% Options for packages loaded elsewhere
\PassOptionsToPackage{unicode}{hyperref}
\PassOptionsToPackage{hyphens}{url}
\PassOptionsToPackage{dvipsnames,svgnames,x11names}{xcolor}
%
\documentclass[
  12pt,
  letterpaper,
  DIV=11,
  numbers=noendperiod]{scrartcl}

\usepackage{amsmath,amssymb}
\usepackage{iftex}
\ifPDFTeX
  \usepackage[T1]{fontenc}
  \usepackage[utf8]{inputenc}
  \usepackage{textcomp} % provide euro and other symbols
\else % if luatex or xetex
  \usepackage{unicode-math}
  \defaultfontfeatures{Scale=MatchLowercase}
  \defaultfontfeatures[\rmfamily]{Ligatures=TeX,Scale=1}
\fi
\usepackage{lmodern}
\ifPDFTeX\else  
    % xetex/luatex font selection
\fi
% Use upquote if available, for straight quotes in verbatim environments
\IfFileExists{upquote.sty}{\usepackage{upquote}}{}
\IfFileExists{microtype.sty}{% use microtype if available
  \usepackage[]{microtype}
  \UseMicrotypeSet[protrusion]{basicmath} % disable protrusion for tt fonts
}{}
\makeatletter
\@ifundefined{KOMAClassName}{% if non-KOMA class
  \IfFileExists{parskip.sty}{%
    \usepackage{parskip}
  }{% else
    \setlength{\parindent}{0pt}
    \setlength{\parskip}{6pt plus 2pt minus 1pt}}
}{% if KOMA class
  \KOMAoptions{parskip=half}}
\makeatother
\usepackage{xcolor}
\usepackage[lmargin=1.5in,rmargin=1.5in,tmargin=1.2in,bmargin=1.2in]{geometry}
\setlength{\emergencystretch}{3em} % prevent overfull lines
\setcounter{secnumdepth}{-\maxdimen} % remove section numbering
% Make \paragraph and \subparagraph free-standing
\ifx\paragraph\undefined\else
  \let\oldparagraph\paragraph
  \renewcommand{\paragraph}[1]{\oldparagraph{#1}\mbox{}}
\fi
\ifx\subparagraph\undefined\else
  \let\oldsubparagraph\subparagraph
  \renewcommand{\subparagraph}[1]{\oldsubparagraph{#1}\mbox{}}
\fi


\providecommand{\tightlist}{%
  \setlength{\itemsep}{0pt}\setlength{\parskip}{0pt}}\usepackage{longtable,booktabs,array}
\usepackage{calc} % for calculating minipage widths
% Correct order of tables after \paragraph or \subparagraph
\usepackage{etoolbox}
\makeatletter
\patchcmd\longtable{\par}{\if@noskipsec\mbox{}\fi\par}{}{}
\makeatother
% Allow footnotes in longtable head/foot
\IfFileExists{footnotehyper.sty}{\usepackage{footnotehyper}}{\usepackage{footnote}}
\makesavenoteenv{longtable}
\usepackage{graphicx}
\makeatletter
\def\maxwidth{\ifdim\Gin@nat@width>\linewidth\linewidth\else\Gin@nat@width\fi}
\def\maxheight{\ifdim\Gin@nat@height>\textheight\textheight\else\Gin@nat@height\fi}
\makeatother
% Scale images if necessary, so that they will not overflow the page
% margins by default, and it is still possible to overwrite the defaults
% using explicit options in \includegraphics[width, height, ...]{}
\setkeys{Gin}{width=\maxwidth,height=\maxheight,keepaspectratio}
% Set default figure placement to htbp
\makeatletter
\def\fps@figure{htbp}
\makeatother
\newlength{\cslhangindent}
\setlength{\cslhangindent}{1.5em}
\newlength{\csllabelwidth}
\setlength{\csllabelwidth}{3em}
\newlength{\cslentryspacingunit} % times entry-spacing
\setlength{\cslentryspacingunit}{\parskip}
\newenvironment{CSLReferences}[2] % #1 hanging-ident, #2 entry spacing
 {% don't indent paragraphs
  \setlength{\parindent}{0pt}
  % turn on hanging indent if param 1 is 1
  \ifodd #1
  \let\oldpar\par
  \def\par{\hangindent=\cslhangindent\oldpar}
  \fi
  % set entry spacing
  \setlength{\parskip}{#2\cslentryspacingunit}
 }%
 {}
\usepackage{calc}
\newcommand{\CSLBlock}[1]{#1\hfill\break}
\newcommand{\CSLLeftMargin}[1]{\parbox[t]{\csllabelwidth}{#1}}
\newcommand{\CSLRightInline}[1]{\parbox[t]{\linewidth - \csllabelwidth}{#1}\break}
\newcommand{\CSLIndent}[1]{\hspace{\cslhangindent}#1}

\KOMAoption{captions}{tableheading}
\usepackage{float} \floatplacement{table}{H}
\makeatletter
\makeatother
\makeatletter
\makeatother
\makeatletter
\@ifpackageloaded{caption}{}{\usepackage{caption}}
\AtBeginDocument{%
\ifdefined\contentsname
  \renewcommand*\contentsname{Table of contents}
\else
  \newcommand\contentsname{Table of contents}
\fi
\ifdefined\listfigurename
  \renewcommand*\listfigurename{List of Figures}
\else
  \newcommand\listfigurename{List of Figures}
\fi
\ifdefined\listtablename
  \renewcommand*\listtablename{List of Tables}
\else
  \newcommand\listtablename{List of Tables}
\fi
\ifdefined\figurename
  \renewcommand*\figurename{Figure}
\else
  \newcommand\figurename{Figure}
\fi
\ifdefined\tablename
  \renewcommand*\tablename{Table}
\else
  \newcommand\tablename{Table}
\fi
}
\@ifpackageloaded{float}{}{\usepackage{float}}
\floatstyle{ruled}
\@ifundefined{c@chapter}{\newfloat{codelisting}{h}{lop}}{\newfloat{codelisting}{h}{lop}[chapter]}
\floatname{codelisting}{Listing}
\newcommand*\listoflistings{\listof{codelisting}{List of Listings}}
\makeatother
\makeatletter
\@ifpackageloaded{caption}{}{\usepackage{caption}}
\@ifpackageloaded{subcaption}{}{\usepackage{subcaption}}
\makeatother
\makeatletter
\@ifpackageloaded{tcolorbox}{}{\usepackage[skins,breakable]{tcolorbox}}
\makeatother
\makeatletter
\@ifundefined{shadecolor}{\definecolor{shadecolor}{rgb}{.97, .97, .97}}
\makeatother
\makeatletter
\makeatother
\makeatletter
\makeatother
\ifLuaTeX
  \usepackage{selnolig}  % disable illegal ligatures
\fi
\IfFileExists{bookmark.sty}{\usepackage{bookmark}}{\usepackage{hyperref}}
\IfFileExists{xurl.sty}{\usepackage{xurl}}{} % add URL line breaks if available
\urlstyle{same} % disable monospaced font for URLs
\hypersetup{
  pdftitle={Convergence Towards Greater Variance},
  pdfauthor={Ezeriki Emetonjor},
  colorlinks=true,
  linkcolor={blue},
  filecolor={Maroon},
  citecolor={Blue},
  urlcolor={Blue},
  pdfcreator={LaTeX via pandoc}}

\title{Convergence Towards Greater Variance}
\author{Ezeriki Emetonjor}
\date{}

\begin{document}
\maketitle
\begin{abstract}
This research investigates heterogeneous effects within a political
exchange paradigm operative in Chinese local political frameworks
involving Provincial Party Secretaries (PPS) and City Mayors. Expanding
on previous investigations that showed how this political exchange
affects higher infrastructure spending in China relative to other
countries, this study specifically centers on subway infrastructure as a
conduit facilitating the alignment of objectives between PPS and city
mayors, thereby augmenting their prospects for future career
advancements. My findings reveal statistically significant differences
in the efficacy of securing subway project approvals in relation to
tenure, which can greatly influence the promotion likelihood of city
mayors in China. Furthermore, my study confirms that there are no
statistically significant differences based on age. git\_personal access
token: ghp\_TtxPEOIrbH1jnb0H1AvsgKfPboQwkf4PZe5t
\end{abstract}
\ifdefined\Shaded\renewenvironment{Shaded}{\begin{tcolorbox}[breakable, borderline west={3pt}{0pt}{shadecolor}, enhanced, interior hidden, frame hidden, boxrule=0pt, sharp corners]}{\end{tcolorbox}}\fi

\newpage{}

\hypertarget{introduction}{%
\section{Introduction}\label{introduction}}

Many developing countries around the world, especially in Africa, Latin
America, and Southeast Asia, have experienced rapid economic growth in
recent years. In particular, countries in sub-Saharan Africa have
experienced growth that surpassed their initial growth rates around the
1960s - 70s, a period in which many of those countries gained
Independence (Diao, McMillan, \& Rodrick, 2019). However, numerous
problems still hold the region back, one of which is an employment
issue. A striking statistic is that for every 10 - 12 million youths
that enter the labor market each year, only 3 million formal sector jobs
are created for them (Fields, 2021). This issue is especially salient,
given the continent's large youth population, with the median age in the
continent being 19 years (IMF Picture this).

It is therefore a serious research question of how best to harness the
budding potential of African countries, particularly in terms of
allocating their numerous resources to productive parts of their
economies in ways that can ensure sustained growth. In this paper, I aim
to contribute to the Literature by analyzing firm level data for four
sub-saharan African countries, namely Nigeria, South Africa, Senegal,
and Malawi??, specifically by characterizing what type of firms, in
terms of size and the estimation of labor productivity, are actually
increasing employment and therefore playing a role in reducing the issue
highlighted above; I aim to focus specifically on the role that the exit
and entry of firms play in determining labor productivity, especially
how it relates to affecting the dispersion of productivity across firms.

Past research has extensively studied the concept of structural
transformation, which is a topic that directly relates to the
reallocation of factors of production, such as labor, to more productive
sectors of the economy; such as (cite specific papers, kuznet, arthur
lewis). There has also been much study on the macroeconomic concept of
Total factor Productivity (TFP), particularly on the role that
mis-allocation, and thus reallocation of factors of production, can play
in increasing TFP growth (cite Hseih \& Klenow, Bailey et al). My
contribution to these broad areas of research is synthesizing and
combining their findings in order to apply to recently compiled data,
specifically on a micro firm level, from which I can get more granular
insights as to the specific mechanisms that could potentially be driving
macro trends as it relates to the allocation of labor in the context of
developing countries.

A recent paper by Diao et al is the main predecessor to my thesis, as
they also explored the dynamics of manufacturing firm size, labor
productivity, employment growth, as well as capital stock with a micro
lens, using firm level data from Ethiopia and Tanzania. Their findings
were consistent with past work that has shown that even though many
African countries have experienced much structural change, by moving
labor out of Agriculture (a sector that has generally been shown as less
productive than modern sectors, such as manufacturing), those same
African countries still record low productivity within the modern sector
that labor is moving to, which is an opposite play-out of how such
change occurred during the growth path for Asian economies (cite Diao,
McMillan, Rodrick, 2019). This, in turn, has not contributed positively
to aggregate productivity growth within those countries. This paper
serves as a great inspiration for this report, as I can now contribute
to the conversation by analyzing if such dynamics exist in other African
countries, particularly in other regions, and if they do not, what
special conditions are at play in those countries. In addition, I am
able to use recently updated data, such as those used in the original
paper, as well as firm survey data compiled by the World Bank to confirm
the macro trends they found in their paper and to put more of a
spotlight on the role that the churning of firms plays.

Although a sizable portion of the Literature has focused on the
Manufacturing sector, specifically due to its ``almost'' traditional
pathway to economic prosperity, based on the experiences of Europe
during the Industrial Revolution or the East Asian Miracle, there's is
another relevant question of if Manufacturing is still a viable path for
upcoming economies. Work by Amirapu and Subramanian (2015) and Fan,
Peters, and Zilibotti (2023) have explored the major role that the
services industry has played in the development of India, a currently
developing country. Therefore, although I aim to focus on the
manufacturing sector for the purposes of this paper, there is still the
underlying importance of other modern sectors, from which this study
could be extended to and which I will return to later in the paper.

The rest of the proposal will proceed as follows: first, I will go
through a Literature review where I highlight important works, that have
inspired this these, followed by a discussion of the data that will be
used and of the methodologies that will be employed

\hypertarget{literature-review}{%
\section{Literature Review}\label{literature-review}}

Many developing countries around the world, especially in Africa, Latin
America, and Southeast Asia, have experienced sustained and rapid
economic growth in recent years. In particular, countries in sub-saharan
Africa have experienced growth that surpassed their initial growth rates
around the 1960s - 70s, a period in which many of those countries gained
Independence (Diao, McMillan, \& Rodrick, 2019). A big driver of this
growth has been the reallocation of factors of production from low
productivity sectors, a typical example being traditional agricultural
activities, to high productivity sectors, such as modern manufacturing,
a process commonly termed Structural Change (Diao, McMillan, \& Rodrick
(2019), McMillan \& Rodrick (2011)). This distinction between high and
low productivity sectors has been explored through models posited by
economists, such as Arthur Lewis the Nobel Laureate, in which there is a
capitalist sector that utilizes productive capital and is an engine for
overall economic growth and the subsistence sector that does not use any
capital but holds ``unlimited amounts of labor'' and can therefore drag
down overall growth and also allows for the existence of a dual economy
(Lewis 1954). (finish reading paper). Also check out kuznets (talk more
here about the equalizing of MPL across sectors ina neoclassical sector
with no distortions or wedges)

However, is structural change enough for sustainable growth? McMillan \&
Rodrick state that it is possible for a country to experience positive
economic growth just from the reallocation of factors of production from
low to high productivity sectors without any productivity growth within
sectors (2011). However, this form of growth might not be sustainable.
Diao, McMillan, \& Rodrick develop a model, which is consistent with
data sources compiled from the World Bank\ldots(get the data sources),
that decomposes aggregate labor productivity growth into a collaboration
between structural change and within sector productivity growth (2019).
The model shows that structural change is the main driver of economic
growth in many African countries and that this growth in particular is
demand driven (Diao, McMillan, \& Rodrick 2019). This specifically means
that much of the labor leaving the traditional agricultural sector has
been due to the cropping up of many modern, small, and many times, less
productive modern firms providing employment opportunities so as to meet
the increased demand of modern products nationally, brought about by
factors, such as foreign aid transfers or increased productivity in the
agricultural sector itself; as a result, the analysis is also consistent
with limited or almost zero productivity growth within the modern
sectors (non-agricultural sectors) (Diao, McMillan, \& Rodrick, 2019).
This is in comparison to development case studies in East Asia whose
growth, characterized by a positive supply shock to the modern sector,
sees a positive contribution of both structural change and within sector
productivity (Diao, McMillan, \& Rodrick, 2019). From this, we can see
that structural change is only part of the development story for
sub-saharan Africa. Another dimension involves the neoclassical Solow
model, which emphasizes the savings rate, the accumulation of capital,
both physical and human, as well as innovation through an endogenous
technological factor; this part of the story focuses more on growth that
assumes a single sector economy (Diao, McMillan, \& Rodrick, (2019)
Fields, Gary). To tie these views together, it might be possible to
think of the structural growth view as more of a reallocation of
existing factors of production across sectors, while the neoclassical
Solow approach as a focus on the growth process within each sector
(Diao, McMillan, \& Rodrick, 2019).

As highlighted above, even though the structural change path is alive
and well within sub-saharan Africa, it seems this path is not directing
surplus labor to the productive parts of the modern sectors, and
providing further boost to productivity growth, based on Diao et al's
model. This is a cause of concern, as successful case studies in
Botswana and Ghana, also analyzed through the lens of the model, both
highlight the necessity of sectoral productivity growth for sustained
aggregate productivity growth; without it, much of the recent growth
accelerations in African countries might be short lived.

Building off of this, it is paramount to understand why this path
diverges into a situation of misallocation between less productive and
more productive firms. Within the literature of misallocation, the
landmark paper by Hseih and Klenow highlights the importance of
re-allocation of resources being a source of huge increases of the
aggregate productivity of China and India with growth rates of 30 -
60\%, using the United States as a benchmark, a relatively un-distorted
economy (2009). Other research, such as that by Bailey et al, has also
emphasized the great contribution of increasing output shares of high
productivity firms and decreasing output shares of low productivity
firms to total factor productivity growth in the U.S. manufacturing
sector (1992). From this, we can see how pivotal re-allocation of
resources is for increased aggregate productivity growth.

One step towards tackling this issue is first identifying possible
reasons for this misallocation. Diao et al take an amazing approach by
analyzing firm level data in Ethiopia and Tanzania, differentiating the
firms by size and estimating the labor productivity and employment
growth of both small, medium, and large firms (2021). Building off the
model established in the earlier paper by Diao, McMillan, \& Rodrick
(2019), Ethiopia and Tanzania serve as great case studies for
understanding this misallocation, as the paper reports that even though
both countries have experienced strong structural change, they have also
experienced weak within sector productivity in the modern sector. Diao
et al find that smaller, less productive firms in both countries have
had much higher employment growth than highly productive firms, which,
in turn, has a negative effect on aggregate labor productivity (2021).
Diao et al posit that a direct cause for this phenomenon is due to the
fact that the highly productive firms are extremely capital intensive,
with capital stock levels that are on par with richer countries, such as
the Czech Republic, and that such technology advanced firms just do not
require as much labor as past industrialization movements have had, such
as in Europe or East Asia (2021).(can also bring up point of
manufacturing now requiring more tech worldwide and GVCs causing a
homogenizing of technology effect)

An interesting question arises, which is the main purpose of this
thesis, is does this phenomenon also plays out in other sub-saharan
African countries. Specifically, I aim to study the following countries:
Nigeria, South Africa, and Senegal (or Malawi or Kenya or Ghana).
Senegal serves as a great comparison to Ethiopia and Tanzania, as it has
also experienced a similar phenomenon of high structural transformation
with low modern sector productivity, particularly for the sectors that
contributed the most to structural change (Diao, McMillan, \& Rodrick,
2019). Nigeria and South Africa, on the other hand, took membership in
another group of sub-saharan African countries who actually experienced
weak structural change, still with a negative correlation with sectoral
productivity growth in the manufacturing sector (Diao, McMillan, \&
Rodrick, 2019). In addition to estimating the labor productivity and
employment growth across time for these countries n a firm level basis,
as done for Ethiopia and Tanzania, I plan to dig deeper into the role
that entry and exit of firms plays in determining labor productivity
growth, specifically using decomposition equations and measurements of
factor productivity used in prior literature (put note in, citing Hsieh
\& Klenow, Bailey et al). I will be using similar data from Diao,
McMillan, \& Rodrick (2019) work, which is data from Groningen Growth
and Development Center (GGDC) (now updated to the latest 2023 version),
to replicate and confirm similar patterns of negative correlation
between structural change and sectoral labor productivity in terms of
their contribution to productivity growth. However, for the main
analysis of the firm level environment for thee countries I will be
using the carefully curated World Bank Enterprise survey panel data,
which I will go into more detail in upcoming sections, that contains
random samples of formal private sector firms, designed to be
representative of the population of firms within each economy.

In the next section, I will go through a description of the World Bank
Enterprise Data and cleaning of the data for my research purposes, as
well go more in depth as to the measurements of productivity and the
identification of the entry and exit of firms

\hypertarget{data-and-methodology}{%
\section{Data and Methodology}\label{data-and-methodology}}

\hypertarget{data}{%
\subsection{Data}\label{data}}

The data that will be utilized in this analysis is provided by the World
Bank through their Formal Private Enterprise survey. The survey is a
random sample of firms, carefully curated to be representative of the
population of firms within each economy; Table 1 highlights some
important statistics for the economies that will be analysed in this
paper. The main sectors of interest, which most of the firms are broadly
categorized into are Manufacturing and Services (cite World Bank). Based
on the focus on the manufacturing sector in past Literature, as
explained earlier, the main focus of this thesis will be on the
distribution of firm productivity within the manufacturing sector. The
data is collected through stratified random sampling, in which the
levels of strata include firm size, sector, and region, and sampling
weights are provided in order to adjust for population representation
(cite World Bank).

(talk about the efforts of the world bank to maintain the panel, cite
that churning world bank paper)

Specifically, the countries of interest for this paper: Senegal, South
Africa, and Nigeria, the years over which the data spans ranges from
2003 to 208, with South Africa having the longest time span between 2003
and 2018. All countries have three years of panel data as decomposed in
table 1. It might therefore be worthwhile to estimate the firm entry and
exit both between the longer time frame between the earliest year and
latest year, as well as the middle year and latest year.

(talk more about the variables of interest, such as labor, sales,
profit, value added, share of sales, sampling weights, control that can
be put in such as strata levels)

The below section goes into more detail as to how I intend to define
firm entry and exit, based on models developed in past Literature, as
well as how I intend to estimate the effect of firm churn on the
variation in labor productivity

\hypertarget{conclusion}{%
\section{Conclusion}\label{conclusion}}

\newpage{}

\hypertarget{refs}{}
\begin{CSLReferences}{0}{0}
\end{CSLReferences}



\end{document}
